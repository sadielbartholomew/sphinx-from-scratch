%% Generated by Sphinx.
\def\sphinxdocclass{report}
\documentclass[letterpaper,10pt,english]{sphinxmanual}
\ifdefined\pdfpxdimen
   \let\sphinxpxdimen\pdfpxdimen\else\newdimen\sphinxpxdimen
\fi \sphinxpxdimen=.75bp\relax

\PassOptionsToPackage{warn}{textcomp}
\usepackage[utf8]{inputenc}
\ifdefined\DeclareUnicodeCharacter
% support both utf8 and utf8x syntaxes
  \ifdefined\DeclareUnicodeCharacterAsOptional
    \def\sphinxDUC#1{\DeclareUnicodeCharacter{"#1}}
  \else
    \let\sphinxDUC\DeclareUnicodeCharacter
  \fi
  \sphinxDUC{00A0}{\nobreakspace}
  \sphinxDUC{2500}{\sphinxunichar{2500}}
  \sphinxDUC{2502}{\sphinxunichar{2502}}
  \sphinxDUC{2514}{\sphinxunichar{2514}}
  \sphinxDUC{251C}{\sphinxunichar{251C}}
  \sphinxDUC{2572}{\textbackslash}
\fi
\usepackage{cmap}
\usepackage[T1]{fontenc}
\usepackage{amsmath,amssymb,amstext}
\usepackage{babel}



\usepackage{times}
\expandafter\ifx\csname T@LGR\endcsname\relax
\else
% LGR was declared as font encoding
  \substitutefont{LGR}{\rmdefault}{cmr}
  \substitutefont{LGR}{\sfdefault}{cmss}
  \substitutefont{LGR}{\ttdefault}{cmtt}
\fi
\expandafter\ifx\csname T@X2\endcsname\relax
  \expandafter\ifx\csname T@T2A\endcsname\relax
  \else
  % T2A was declared as font encoding
    \substitutefont{T2A}{\rmdefault}{cmr}
    \substitutefont{T2A}{\sfdefault}{cmss}
    \substitutefont{T2A}{\ttdefault}{cmtt}
  \fi
\else
% X2 was declared as font encoding
  \substitutefont{X2}{\rmdefault}{cmr}
  \substitutefont{X2}{\sfdefault}{cmss}
  \substitutefont{X2}{\ttdefault}{cmtt}
\fi


\usepackage[Bjarne]{fncychap}
\usepackage{sphinx}

\fvset{fontsize=\small}
\usepackage{geometry}


% Include hyperref last.
\usepackage{hyperref}
% Fix anchor placement for figures with captions.
\usepackage{hypcap}% it must be loaded after hyperref.
% Set up styles of URL: it should be placed after hyperref.
\urlstyle{same}

\addto\captionsenglish{\renewcommand{\contentsname}{Contents:}}

\usepackage{sphinxmessages}
\setcounter{tocdepth}{1}



\title{Quadrilaterals}
\date{Nov 11, 2020}
\release{1.0}
\author{Sadie}
\newcommand{\sphinxlogo}{\vbox{}}
\renewcommand{\releasename}{Release}
\makeindex
\begin{document}

\pagestyle{empty}
\sphinxmaketitle
\pagestyle{plain}
\sphinxtableofcontents
\pagestyle{normal}
\phantomsection\label{\detokenize{index::doc}}



\chapter{Configuring Sphinx from scratch}
\label{\detokenize{context:configuring-sphinx-from-scratch}}\label{\detokenize{context::doc}}
Version 1.0

\sphinxstylestrong{A software demonstration by Sadie Bartholomew of the Sphinx documentation
generator.}

\begin{sphinxadmonition}{note}{Note:}
I am \sphinxstyleemphasis{not associated with the Sphinx project at all}. I have chosen
to give a demo of Sphinx because it is a tool I have found extremely
useful in my work in RSE roles.
\end{sphinxadmonition}


\section{Context}
\label{\detokenize{context:context}}

\subsection{Demo format}
\label{\detokenize{context:demo-format}}
This is a demonstration so will be largely conducted in the terminal.
However, notes are provided, as incoporated into the example documentation
itself.

\begin{sphinxadmonition}{tip}{Tip:}
The source and built documentation, which includes these notes,
is (and will permanently remain) hosted on GitHub at:
\sphinxurl{https://github.com/sadielbartholomew/sphinx-from-scratch}
\end{sphinxadmonition}


\subsection{The Dummy Project}
\label{\detokenize{context:the-dummy-project}}
\sphinxcode{\sphinxupquote{quadrilaterals}}: a very simple and trivial object\sphinxhyphen{}oriented Python codebase
used as a placeholder for a real\sphinxhyphen{}life and very likely more complex project.

The codebase models categories of two\sphinxhyphen{}dimensional four\sphinxhyphen{}sided shape, which
are collectively called quadrilaterals. For each category, such as a
square or a rhombus, there are methods to calculate the area, perimeter and
number of axes of symmetry.

A diagram %
\begin{footnote}[1]\sphinxAtStartFootnote
Image sourced from \sphinxhref{https://www.geogebra.org/m/bm4ja4wb}{this webpage}.
%
\end{footnote} showing the main classes in the dummy project, and
their inheritance hierarchy %
\begin{footnote}[2]\sphinxAtStartFootnote
Strictly this diagram does not capture one other
relationship between the quadrilaterals which also exists, and which the
dummy project incorporates into the class hierarchy, namely that a
\sphinxstyleemphasis{rhombus} is a special form of \sphinxstyleemphasis{kite}.
%
\end{footnote} :

\noindent\sphinxincludegraphics{{material-ncsc3adx}.png}

An example of the code in use (Python console notation):

\begin{sphinxVerbatim}[commandchars=\\\{\}]
\PYG{g+gp}{\PYGZgt{}\PYGZgt{}\PYGZgt{} }\PYG{k+kn}{import} \PYG{n+nn}{quadrilaterals}
\PYG{g+gp}{\PYGZgt{}\PYGZgt{}\PYGZgt{} }\PYG{n}{square} \PYG{o}{=} \PYG{n}{quadrilaterals}\PYG{o}{.}\PYG{n}{Square}\PYG{p}{(}\PYG{l+m+mi}{5}\PYG{p}{)}
\PYG{g+gp}{\PYGZgt{}\PYGZgt{}\PYGZgt{} }\PYG{n}{square}\PYG{o}{.}\PYG{n}{area}\PYG{p}{(}\PYG{p}{)}
\PYG{g+go}{25}
\PYG{g+gp}{\PYGZgt{}\PYGZgt{}\PYGZgt{} }\PYG{n}{square}\PYG{o}{.}\PYG{n}{perimeter}\PYG{p}{(}\PYG{p}{)}
\PYG{g+go}{20}
\PYG{g+gp}{\PYGZgt{}\PYGZgt{}\PYGZgt{} }\PYG{n}{help}\PYG{p}{(}\PYG{n}{square}\PYG{p}{)}
\PYG{g+go}{Help on Square in module quadrilaterals.parallelograms.rectangles.square object:}

\PYG{g+go}{class Square(quadrilaterals.parallelograms.rectangles.rectangle.Rectangle, quadrilaterals.kites.rhombi.rhombus.Rhombus)}
\PYG{g+go}{ |  Square(side\PYGZus{}length)}
\PYG{g+go}{ |}
\PYG{g+go}{ |  Base class common to all squares.}
\PYG{g+go}{ |}
\PYG{g+go}{ |  \PYGZgt{}\PYGZgt{}\PYGZgt{} import quadrilaterals}
\PYG{g+go}{ |  \PYGZgt{}\PYGZgt{}\PYGZgt{} q = quadrilaterals.Square(0.2)}
\PYG{g+go}{ |}
\PYG{g+go}{ |  Method resolution order:}
\PYG{g+go}{ |      Square}
\PYG{g+go}{ |      quadrilaterals.parallelograms.rectangles.rectangle.Rectangle}
\PYG{g+go}{ |      quadrilaterals.kites.rhombi.rhombus.Rhombus}
\PYG{g+go}{ |      quadrilaterals.kites.kite.Kite}
\PYG{g+go}{ |      quadrilaterals.parallelograms.parallelogram.Parallelogram}
\PYG{g+go}{ |      quadrilaterals.quadrilateral.Quadrilateral}
\PYG{g+go}{ |      builtins.object}
\PYG{g+go}{ |}
\PYG{g+gp}{...}
\PYG{g+gp}{...}
\PYG{g+gp}{...}
\PYG{g+gp}{\PYGZgt{}\PYGZgt{}\PYGZgt{} }\PYG{n}{parallelogram} \PYG{o}{=} \PYG{n}{quadrilaterals}\PYG{o}{.}\PYG{n}{Parallelogram}\PYG{p}{(}\PYG{l+m+mi}{5}\PYG{p}{,} \PYG{l+m+mi}{2}\PYG{p}{,} \PYG{l+m+mf}{1.5}\PYG{p}{)}
\PYG{g+gp}{\PYGZgt{}\PYGZgt{}\PYGZgt{} }\PYG{n}{help}\PYG{p}{(}\PYG{n}{parallelogram}\PYG{p}{)}
\PYG{g+go}{Help on Parallelogram in module quadrilaterals.parallelograms.parallelogram object:}

\PYG{g+go}{class Parallelogram(quadrilaterals.quadrilateral.Quadrilateral)}
\PYG{g+go}{ |  Parallelogram(parallel\PYGZus{}side\PYGZus{}A\PYGZus{}length, parallel\PYGZus{}side\PYGZus{}B\PYGZus{}length, any\PYGZus{}angle=None)}
\PYG{g+go}{ |}
\PYG{g+go}{ |  Base class common to all parallelograms.}
\PYG{g+go}{ |}
\PYG{g+go}{ |  \PYGZgt{}\PYGZgt{}\PYGZgt{} import math}
\PYG{g+go}{ |  \PYGZgt{}\PYGZgt{}\PYGZgt{} q1 = quadrilaterals.Parallelogram(2, 5, 1.1)}
\PYG{g+go}{ |  \PYGZgt{}\PYGZgt{}\PYGZgt{} q2 = quadrilaterals.Parallelogram(2, 5, math.pi \PYGZhy{} 1.1)}
\PYG{g+go}{ |}
\PYG{g+go}{ |  Method resolution order:}
\PYG{g+go}{ |      Parallelogram}
\PYG{g+go}{ |      quadrilaterals.quadrilateral.Quadrilateral}
\PYG{g+go}{ |      builtins.object}
\PYG{g+go}{ |}
\PYG{g+gp}{...}
\PYG{g+gp}{...}
\PYG{g+gp}{...}
\PYG{g+gp}{\PYGZgt{}\PYGZgt{}\PYGZgt{} }\PYG{n}{parallelogram}\PYG{o}{.}\PYG{n}{area}\PYG{p}{(}\PYG{p}{)}
\PYG{g+go}{9.974949866040545}
\PYG{g+gp}{\PYGZgt{}\PYGZgt{}\PYGZgt{} }\PYG{n}{parallelogram}\PYG{o}{.}\PYG{n}{perimeter}\PYG{p}{(}\PYG{p}{)}
\PYG{g+go}{14}
\end{sphinxVerbatim}


\section{Quick links}
\label{\detokenize{context:quick-links}}

\subsection{Examples of Sphinx\sphinxhyphen{}generated documentation}
\label{\detokenize{context:examples-of-sphinx-generated-documentation}}
It is hard to tell what has been made with Sphinx without looking at the
codebase source, but often if you scroll to the bottom of some
documentation you can often see a “Created using Sphinx \textless{}sphinx version\textgreater{}”
note in the footer. A small number of examples are referenced below.
\begin{itemize}
\item {} 
A large but absolutely not comprehensive listing collected by the Sphinx
team: \sphinxurl{https://www.sphinx-doc.org/en/master/examples.html}

\item {} 
Sphinx’s own documentation (made with Sphinx, of course!):
\sphinxurl{https://www.sphinx-doc.org/en/master/}

\item {} 
Python 3 documentation: \sphinxurl{https://docs.python.org/3/}

\item {} 
Python 2 documentation: \sphinxurl{https://docs.python.org/3/}

\item {} 
Tornado documentation: \sphinxurl{https://www.tornadoweb.org/en/stable/}

\item {} 
Official RST documentation (note it is also made in Sphinx):
\sphinxurl{https://docutils.readthedocs.io/en/sphinx-docs/user/rst/quickstart.html}

\item {} 
Dask documentation: \sphinxurl{https://docs.dask.org/en/latest/}

\item {} 
JupyterHub documentation: \sphinxurl{https://jupyterhub.readthedocs.io/en/stable/}

\item {} 
NumPy documentation: \sphinxurl{https://numpy.org/doc/stable/reference/}

\item {} 
A small (slightly old) listing of non\sphinxhyphen{}Python projects with Sphinx
documentation:
\sphinxurl{https://ericholscher.com/blog/2014/feb/11/sphinx-isnt-just-for-python/}

\item {} 
An example of a book made with Sphinx: \sphinxurl{https://www.theoretical-physics.net/}

\end{itemize}


\subsection{Sphinx}
\label{\detokenize{context:sphinx}}

\subsubsection{Basic}
\label{\detokenize{context:basic}}\begin{itemize}
\item {} 
Sphinx documentation homepage: \sphinxurl{https://www.sphinx-doc.org/en/master/}

\item {} 
Quick start including \sphinxtitleref{sphinx\sphinxhyphen{}quickstart} command:
\sphinxurl{https://www.sphinx-doc.org/en/master/usage/quickstart.html}

\item {} 
HTML themes: \sphinxurl{https://www.sphinx-doc.org/en/master/usage/theming.html}

\end{itemize}


\subsubsection{Advanced}
\label{\detokenize{context:advanced}}\begin{itemize}
\item {} 
Guidance on extensions:
\sphinxurl{https://www.sphinx-doc.org/en/master/usage/extensions/index.html}

\item {} 
An “awesome” listing of extra Sphinx resources:
\sphinxurl{https://github.com/yoloseem/awesome-sphinxdoc}

\item {} 
Big listing of even more HTML themes: \sphinxurl{https://sphinx-themes.org/}

\item {} 
Our customisation of the “alabaster” default theme, in practice:
\sphinxurl{https://ncas-cms.github.io/cf-python/}

\item {} 
Example Sphinx extension project repositories:
\begin{itemize}
\item {} 
\sphinxcode{\sphinxupquote{sphinx\sphinxhyphen{}copybutton}} (button to copy code in code examples):
\sphinxurl{https://github.com/executablebooks/sphinx-copybutton}

\item {} 
\sphinxcode{\sphinxupquote{sphinx\sphinxhyphen{}toggleprompt}} (button to hide prompts and outputs in
console\sphinxhyphen{}like code examples):
\sphinxurl{https://github.com/jurasofish/sphinx-toggleprompt}

\item {} 
\sphinxcode{\sphinxupquote{sphinx\sphinxhyphen{}pyreverse}} (generates UML diagramms of the code or parts of
it into the documentation): \sphinxurl{https://github.com/alendit/sphinx-pyreverse}

\end{itemize}

\end{itemize}


\subsubsection{reStructuredText (\sphinxtitleref{.rst} extension) format}
\label{\detokenize{context:restructuredtext-rst-extension-format}}\begin{itemize}
\item {} 
Official reStructuredText documentation (note it is also made in Sphinx):
\sphinxurl{https://docutils.readthedocs.io/en/sphinx-docs/user/rst/quickstart.html}

\item {} 
Helpful cheatseets:
\begin{itemize}
\item {} 
\sphinxurl{https://docutils.sourceforge.io/docs/user/rst/quickref.html}

\item {} 
\sphinxurl{https://github.com/ralsina/rst-cheatsheet/blob/master/rst-cheatsheet.rst}

\end{itemize}

\end{itemize}


\subsubsection{Documenting projects with Sphinx}
\label{\detokenize{context:documenting-projects-with-sphinx}}\begin{itemize}
\item {} 
A nice blog post: \sphinxurl{https://medium.com/@richdayandnight/a-simple-tutorial-on-how-to-document-your-python-project-using-sphinx-and-rinohtype-177c22a15b5b}

\item {} 
A great post showing how to use Sphinx with other tools to document a C++
project: \sphinxurl{https://devblogs.microsoft.com/cppblog/clear-functional-c-documentation-with-sphinx-breathe-doxygen-cmake/}

\end{itemize}


\chapter{API}
\label{\detokenize{api_reference:api}}\label{\detokenize{api_reference::doc}}

\begin{savenotes}\sphinxatlongtablestart\begin{longtable}[c]{\X{1}{2}\X{1}{2}}
\hline

\endfirsthead

\multicolumn{2}{c}%
{\makebox[0pt]{\sphinxtablecontinued{\tablename\ \thetable{} \textendash{} continued from previous page}}}\\
\hline

\endhead

\hline
\multicolumn{2}{r}{\makebox[0pt][r]{\sphinxtablecontinued{continues on next page}}}\\
\endfoot

\endlastfoot

{\hyperref[\detokenize{_autosummary/quadrilaterals.quadrilateral.Quadrilateral:quadrilaterals.quadrilateral.Quadrilateral}]{\sphinxcrossref{\sphinxcode{\sphinxupquote{quadrilateral.Quadrilateral}}}}}(side\_A\_length, …)
&
Base class common to all quadrilaterals, four\sphinxhyphen{}sided polygons.
\\
\hline
{\hyperref[\detokenize{_autosummary/quadrilaterals.kites.kite.Kite:quadrilaterals.kites.kite.Kite}]{\sphinxcrossref{\sphinxcode{\sphinxupquote{kites.kite.Kite}}}}}(side\_A\_length, side\_B\_length)
&
Base class common to all kites.
\\
\hline
{\hyperref[\detokenize{_autosummary/quadrilaterals.kites.rhombi.rhombus.Rhombus:quadrilaterals.kites.rhombi.rhombus.Rhombus}]{\sphinxcrossref{\sphinxcode{\sphinxupquote{kites.rhombi.rhombus.Rhombus}}}}}(side\_length{[}, …{]})
&
Base class common to all rhombi.
\\
\hline
{\hyperref[\detokenize{_autosummary/quadrilaterals.parallelograms.parallelogram.Parallelogram:quadrilaterals.parallelograms.parallelogram.Parallelogram}]{\sphinxcrossref{\sphinxcode{\sphinxupquote{parallelograms.parallelogram.Parallelogram}}}}}(…)
&
Base class common to all parallelograms.
\\
\hline
{\hyperref[\detokenize{_autosummary/quadrilaterals.parallelograms.rectangles.rectangle.Rectangle:quadrilaterals.parallelograms.rectangles.rectangle.Rectangle}]{\sphinxcrossref{\sphinxcode{\sphinxupquote{parallelograms.rectangles.rectangle.Rectangle}}}}}(…)
&
Base class common to all rectangles.
\\
\hline
{\hyperref[\detokenize{_autosummary/quadrilaterals.parallelograms.rectangles.square.Square:quadrilaterals.parallelograms.rectangles.square.Square}]{\sphinxcrossref{\sphinxcode{\sphinxupquote{parallelograms.rectangles.square.Square}}}}}(…)
&
Base class common to all squares.
\\
\hline
{\hyperref[\detokenize{_autosummary/quadrilaterals.trapezia.trapezium.Trapezium:quadrilaterals.trapezia.trapezium.Trapezium}]{\sphinxcrossref{\sphinxcode{\sphinxupquote{trapezia.trapezium.Trapezium}}}}}(…{[}, height{]})
&
Base class common to all trapezia.
\\
\hline
{\hyperref[\detokenize{_autosummary/quadrilaterals.trapezia.isosceles_trapezia.isosceles_trapezium.IsoscelesTrapezium:quadrilaterals.trapezia.isosceles_trapezia.isosceles_trapezium.IsoscelesTrapezium}]{\sphinxcrossref{\sphinxcode{\sphinxupquote{trapezia.isosceles\_trapezia.isosceles\_trapezium.IsoscelesTrapezium}}}}}(…)
&
Base class common to all isosceles trapezia.
\\
\hline
\end{longtable}\sphinxatlongtableend\end{savenotes}


\section{quadrilaterals.quadrilateral.Quadrilateral}
\label{\detokenize{_autosummary/quadrilaterals.quadrilateral.Quadrilateral:quadrilaterals-quadrilateral-quadrilateral}}\label{\detokenize{_autosummary/quadrilaterals.quadrilateral.Quadrilateral::doc}}\index{Quadrilateral (class in quadrilaterals.quadrilateral)@\spxentry{Quadrilateral}\spxextra{class in quadrilaterals.quadrilateral}}

\begin{fulllineitems}
\phantomsection\label{\detokenize{_autosummary/quadrilaterals.quadrilateral.Quadrilateral:quadrilaterals.quadrilateral.Quadrilateral}}\pysiglinewithargsret{\sphinxbfcode{\sphinxupquote{class }}\sphinxcode{\sphinxupquote{quadrilaterals.quadrilateral.}}\sphinxbfcode{\sphinxupquote{Quadrilateral}}}{\emph{\DUrole{n}{side\_A\_length}}, \emph{\DUrole{n}{side\_B\_length}}, \emph{\DUrole{n}{side\_C\_length}}, \emph{\DUrole{n}{side\_D\_length}}}{}
Base class common to all quadrilaterals, four\sphinxhyphen{}sided polygons.

\begin{sphinxVerbatim}[commandchars=\\\{\}]
\PYG{g+gp}{\PYGZgt{}\PYGZgt{}\PYGZgt{} }\PYG{n}{q} \PYG{o}{=} \PYG{n}{quadrilaterals}\PYG{o}{.}\PYG{n}{Quadrilateral}\PYG{p}{(}\PYG{l+m+mi}{1}\PYG{p}{,} \PYG{l+m+mi}{2}\PYG{p}{,} \PYG{l+m+mi}{3}\PYG{p}{,} \PYG{l+m+mi}{4}\PYG{p}{)}
\end{sphinxVerbatim}
\index{\_\_init\_\_() (quadrilaterals.quadrilateral.Quadrilateral method)@\spxentry{\_\_init\_\_()}\spxextra{quadrilaterals.quadrilateral.Quadrilateral method}}

\begin{fulllineitems}
\phantomsection\label{\detokenize{_autosummary/quadrilaterals.quadrilateral.Quadrilateral:quadrilaterals.quadrilateral.Quadrilateral.__init__}}\pysiglinewithargsret{\sphinxbfcode{\sphinxupquote{\_\_init\_\_}}}{\emph{\DUrole{n}{side\_A\_length}}, \emph{\DUrole{n}{side\_B\_length}}, \emph{\DUrole{n}{side\_C\_length}}, \emph{\DUrole{n}{side\_D\_length}}}{}
Initialize self.  See help(type(self)) for accurate signature.

\end{fulllineitems}

\subsubsection*{Methods}


\begin{savenotes}\sphinxatlongtablestart\begin{longtable}[c]{\X{1}{2}\X{1}{2}}
\hline

\endfirsthead

\multicolumn{2}{c}%
{\makebox[0pt]{\sphinxtablecontinued{\tablename\ \thetable{} \textendash{} continued from previous page}}}\\
\hline

\endhead

\hline
\multicolumn{2}{r}{\makebox[0pt][r]{\sphinxtablecontinued{continues on next page}}}\\
\endfoot

\endlastfoot

{\hyperref[\detokenize{_autosummary/quadrilaterals.quadrilateral.Quadrilateral:quadrilaterals.quadrilateral.Quadrilateral.__init__}]{\sphinxcrossref{\sphinxcode{\sphinxupquote{\_\_init\_\_}}}}}(side\_A\_length, side\_B\_length, …)
&
Initialize self.
\\
\hline
\sphinxcode{\sphinxupquote{area}}()
&
Return the area of the polygon.
\\
\hline
\sphinxcode{\sphinxupquote{axes\_of\_symmetry}}()
&
Return the number of axes of symmetry of the quadrilateral.
\\
\hline
\sphinxcode{\sphinxupquote{perimeter}}()
&
Return the perimeter of the quadrilateral.
\\
\hline
\end{longtable}\sphinxatlongtableend\end{savenotes}
\subsubsection*{Attributes}


\begin{savenotes}\sphinxatlongtablestart\begin{longtable}[c]{\X{1}{2}\X{1}{2}}
\hline

\endfirsthead

\multicolumn{2}{c}%
{\makebox[0pt]{\sphinxtablecontinued{\tablename\ \thetable{} \textendash{} continued from previous page}}}\\
\hline

\endhead

\hline
\multicolumn{2}{r}{\makebox[0pt][r]{\sphinxtablecontinued{continues on next page}}}\\
\endfoot

\endlastfoot

\sphinxcode{\sphinxupquote{dimensions}}
&

\\
\hline
\sphinxcode{\sphinxupquote{sides}}
&

\\
\hline
\end{longtable}\sphinxatlongtableend\end{savenotes}

\end{fulllineitems}



\section{quadrilaterals.kites.kite.Kite}
\label{\detokenize{_autosummary/quadrilaterals.kites.kite.Kite:quadrilaterals-kites-kite-kite}}\label{\detokenize{_autosummary/quadrilaterals.kites.kite.Kite::doc}}\index{Kite (class in quadrilaterals.kites.kite)@\spxentry{Kite}\spxextra{class in quadrilaterals.kites.kite}}

\begin{fulllineitems}
\phantomsection\label{\detokenize{_autosummary/quadrilaterals.kites.kite.Kite:quadrilaterals.kites.kite.Kite}}\pysiglinewithargsret{\sphinxbfcode{\sphinxupquote{class }}\sphinxcode{\sphinxupquote{quadrilaterals.kites.kite.}}\sphinxbfcode{\sphinxupquote{Kite}}}{\emph{\DUrole{n}{side\_A\_length}}, \emph{\DUrole{n}{side\_B\_length}}, \emph{\DUrole{n}{angle\_AB}\DUrole{o}{=}\DUrole{default_value}{None}}}{}~\begin{quote}

Base class common to all kites.
\end{quote}

\begin{sphinxVerbatim}[commandchars=\\\{\}]
\PYG{g+gp}{\PYGZgt{}\PYGZgt{}\PYGZgt{} }\PYG{n}{q} \PYG{o}{=} \PYG{n}{quadrilaterals}\PYG{o}{.}\PYG{n}{Kite}\PYG{p}{(}\PYG{l+m+mi}{3}\PYG{p}{,} \PYG{l+m+mi}{5}\PYG{p}{,} \PYG{l+m+mf}{2.4}\PYG{p}{)}
\end{sphinxVerbatim}
\index{\_\_init\_\_() (quadrilaterals.kites.kite.Kite method)@\spxentry{\_\_init\_\_()}\spxextra{quadrilaterals.kites.kite.Kite method}}

\begin{fulllineitems}
\phantomsection\label{\detokenize{_autosummary/quadrilaterals.kites.kite.Kite:quadrilaterals.kites.kite.Kite.__init__}}\pysiglinewithargsret{\sphinxbfcode{\sphinxupquote{\_\_init\_\_}}}{\emph{\DUrole{n}{side\_A\_length}}, \emph{\DUrole{n}{side\_B\_length}}, \emph{\DUrole{n}{angle\_AB}\DUrole{o}{=}\DUrole{default_value}{None}}}{}
Initialize self.  See help(type(self)) for accurate signature.

\end{fulllineitems}

\subsubsection*{Methods}


\begin{savenotes}\sphinxatlongtablestart\begin{longtable}[c]{\X{1}{2}\X{1}{2}}
\hline

\endfirsthead

\multicolumn{2}{c}%
{\makebox[0pt]{\sphinxtablecontinued{\tablename\ \thetable{} \textendash{} continued from previous page}}}\\
\hline

\endhead

\hline
\multicolumn{2}{r}{\makebox[0pt][r]{\sphinxtablecontinued{continues on next page}}}\\
\endfoot

\endlastfoot

{\hyperref[\detokenize{_autosummary/quadrilaterals.kites.kite.Kite:quadrilaterals.kites.kite.Kite.__init__}]{\sphinxcrossref{\sphinxcode{\sphinxupquote{\_\_init\_\_}}}}}(side\_A\_length, side\_B\_length{[}, …{]})
&
Initialize self.
\\
\hline
\sphinxcode{\sphinxupquote{area}}()
&
Return the area of the kite.
\\
\hline
\sphinxcode{\sphinxupquote{axes\_of\_symmetry}}()
&
Return the number of axes of symmetry of the quadrilateral.
\\
\hline
\sphinxcode{\sphinxupquote{perimeter}}()
&
Return the perimeter of the quadrilateral.
\\
\hline
\end{longtable}\sphinxatlongtableend\end{savenotes}
\subsubsection*{Attributes}


\begin{savenotes}\sphinxatlongtablestart\begin{longtable}[c]{\X{1}{2}\X{1}{2}}
\hline

\endfirsthead

\multicolumn{2}{c}%
{\makebox[0pt]{\sphinxtablecontinued{\tablename\ \thetable{} \textendash{} continued from previous page}}}\\
\hline

\endhead

\hline
\multicolumn{2}{r}{\makebox[0pt][r]{\sphinxtablecontinued{continues on next page}}}\\
\endfoot

\endlastfoot

\sphinxcode{\sphinxupquote{dimensions}}
&

\\
\hline
\sphinxcode{\sphinxupquote{sides}}
&

\\
\hline
\end{longtable}\sphinxatlongtableend\end{savenotes}

\end{fulllineitems}



\section{quadrilaterals.kites.rhombi.rhombus.Rhombus}
\label{\detokenize{_autosummary/quadrilaterals.kites.rhombi.rhombus.Rhombus:quadrilaterals-kites-rhombi-rhombus-rhombus}}\label{\detokenize{_autosummary/quadrilaterals.kites.rhombi.rhombus.Rhombus::doc}}\index{Rhombus (class in quadrilaterals.kites.rhombi.rhombus)@\spxentry{Rhombus}\spxextra{class in quadrilaterals.kites.rhombi.rhombus}}

\begin{fulllineitems}
\phantomsection\label{\detokenize{_autosummary/quadrilaterals.kites.rhombi.rhombus.Rhombus:quadrilaterals.kites.rhombi.rhombus.Rhombus}}\pysiglinewithargsret{\sphinxbfcode{\sphinxupquote{class }}\sphinxcode{\sphinxupquote{quadrilaterals.kites.rhombi.rhombus.}}\sphinxbfcode{\sphinxupquote{Rhombus}}}{\emph{\DUrole{n}{side\_length}}, \emph{\DUrole{n}{any\_angle}\DUrole{o}{=}\DUrole{default_value}{None}}}{}
Base class common to all rhombi.

\begin{sphinxVerbatim}[commandchars=\\\{\}]
\PYG{g+gp}{\PYGZgt{}\PYGZgt{}\PYGZgt{} }\PYG{n}{q} \PYG{o}{=} \PYG{n}{quadrilaterals}\PYG{o}{.}\PYG{n}{Rhombus}\PYG{p}{(}\PYG{l+m+mi}{4}\PYG{p}{,} \PYG{l+m+mf}{1.5}\PYG{p}{)}
\end{sphinxVerbatim}
\index{\_\_init\_\_() (quadrilaterals.kites.rhombi.rhombus.Rhombus method)@\spxentry{\_\_init\_\_()}\spxextra{quadrilaterals.kites.rhombi.rhombus.Rhombus method}}

\begin{fulllineitems}
\phantomsection\label{\detokenize{_autosummary/quadrilaterals.kites.rhombi.rhombus.Rhombus:quadrilaterals.kites.rhombi.rhombus.Rhombus.__init__}}\pysiglinewithargsret{\sphinxbfcode{\sphinxupquote{\_\_init\_\_}}}{\emph{\DUrole{n}{side\_length}}, \emph{\DUrole{n}{any\_angle}\DUrole{o}{=}\DUrole{default_value}{None}}}{}
Initialize self.  See help(type(self)) for accurate signature.

\end{fulllineitems}

\subsubsection*{Methods}


\begin{savenotes}\sphinxatlongtablestart\begin{longtable}[c]{\X{1}{2}\X{1}{2}}
\hline

\endfirsthead

\multicolumn{2}{c}%
{\makebox[0pt]{\sphinxtablecontinued{\tablename\ \thetable{} \textendash{} continued from previous page}}}\\
\hline

\endhead

\hline
\multicolumn{2}{r}{\makebox[0pt][r]{\sphinxtablecontinued{continues on next page}}}\\
\endfoot

\endlastfoot

{\hyperref[\detokenize{_autosummary/quadrilaterals.kites.rhombi.rhombus.Rhombus:quadrilaterals.kites.rhombi.rhombus.Rhombus.__init__}]{\sphinxcrossref{\sphinxcode{\sphinxupquote{\_\_init\_\_}}}}}(side\_length{[}, any\_angle{]})
&
Initialize self.
\\
\hline
\sphinxcode{\sphinxupquote{area}}()
&
Return the area of the kite.
\\
\hline
\sphinxcode{\sphinxupquote{axes\_of\_symmetry}}()
&
Return the number of axes of symmetry of the quadrilateral.
\\
\hline
\sphinxcode{\sphinxupquote{perimeter}}()
&
Return the perimeter of the quadrilateral.
\\
\hline
\end{longtable}\sphinxatlongtableend\end{savenotes}
\subsubsection*{Attributes}


\begin{savenotes}\sphinxatlongtablestart\begin{longtable}[c]{\X{1}{2}\X{1}{2}}
\hline

\endfirsthead

\multicolumn{2}{c}%
{\makebox[0pt]{\sphinxtablecontinued{\tablename\ \thetable{} \textendash{} continued from previous page}}}\\
\hline

\endhead

\hline
\multicolumn{2}{r}{\makebox[0pt][r]{\sphinxtablecontinued{continues on next page}}}\\
\endfoot

\endlastfoot

\sphinxcode{\sphinxupquote{dimensions}}
&

\\
\hline
\sphinxcode{\sphinxupquote{sides}}
&

\\
\hline
\end{longtable}\sphinxatlongtableend\end{savenotes}

\end{fulllineitems}



\section{quadrilaterals.parallelograms.parallelogram.Parallelogram}
\label{\detokenize{_autosummary/quadrilaterals.parallelograms.parallelogram.Parallelogram:quadrilaterals-parallelograms-parallelogram-parallelogram}}\label{\detokenize{_autosummary/quadrilaterals.parallelograms.parallelogram.Parallelogram::doc}}\index{Parallelogram (class in quadrilaterals.parallelograms.parallelogram)@\spxentry{Parallelogram}\spxextra{class in quadrilaterals.parallelograms.parallelogram}}

\begin{fulllineitems}
\phantomsection\label{\detokenize{_autosummary/quadrilaterals.parallelograms.parallelogram.Parallelogram:quadrilaterals.parallelograms.parallelogram.Parallelogram}}\pysiglinewithargsret{\sphinxbfcode{\sphinxupquote{class }}\sphinxcode{\sphinxupquote{quadrilaterals.parallelograms.parallelogram.}}\sphinxbfcode{\sphinxupquote{Parallelogram}}}{\emph{\DUrole{n}{parallel\_side\_A\_length}}, \emph{\DUrole{n}{parallel\_side\_B\_length}}, \emph{\DUrole{n}{any\_angle}\DUrole{o}{=}\DUrole{default_value}{None}}}{}
Base class common to all parallelograms.

\begin{sphinxVerbatim}[commandchars=\\\{\}]
\PYG{g+gp}{\PYGZgt{}\PYGZgt{}\PYGZgt{} }\PYG{k+kn}{import} \PYG{n+nn}{math}
\PYG{g+gp}{\PYGZgt{}\PYGZgt{}\PYGZgt{} }\PYG{n}{q1} \PYG{o}{=} \PYG{n}{quadrilaterals}\PYG{o}{.}\PYG{n}{Parallelogram}\PYG{p}{(}\PYG{l+m+mi}{2}\PYG{p}{,} \PYG{l+m+mi}{5}\PYG{p}{,} \PYG{l+m+mf}{1.1}\PYG{p}{)}
\PYG{g+gp}{\PYGZgt{}\PYGZgt{}\PYGZgt{} }\PYG{n}{q2} \PYG{o}{=} \PYG{n}{quadrilaterals}\PYG{o}{.}\PYG{n}{Parallelogram}\PYG{p}{(}\PYG{l+m+mi}{2}\PYG{p}{,} \PYG{l+m+mi}{5}\PYG{p}{,} \PYG{n}{math}\PYG{o}{.}\PYG{n}{pi} \PYG{o}{\PYGZhy{}} \PYG{l+m+mf}{1.1}\PYG{p}{)}
\end{sphinxVerbatim}
\index{\_\_init\_\_() (quadrilaterals.parallelograms.parallelogram.Parallelogram method)@\spxentry{\_\_init\_\_()}\spxextra{quadrilaterals.parallelograms.parallelogram.Parallelogram method}}

\begin{fulllineitems}
\phantomsection\label{\detokenize{_autosummary/quadrilaterals.parallelograms.parallelogram.Parallelogram:quadrilaterals.parallelograms.parallelogram.Parallelogram.__init__}}\pysiglinewithargsret{\sphinxbfcode{\sphinxupquote{\_\_init\_\_}}}{\emph{\DUrole{n}{parallel\_side\_A\_length}}, \emph{\DUrole{n}{parallel\_side\_B\_length}}, \emph{\DUrole{n}{any\_angle}\DUrole{o}{=}\DUrole{default_value}{None}}}{}
Initialize self.  See help(type(self)) for accurate signature.

\end{fulllineitems}

\subsubsection*{Methods}


\begin{savenotes}\sphinxatlongtablestart\begin{longtable}[c]{\X{1}{2}\X{1}{2}}
\hline

\endfirsthead

\multicolumn{2}{c}%
{\makebox[0pt]{\sphinxtablecontinued{\tablename\ \thetable{} \textendash{} continued from previous page}}}\\
\hline

\endhead

\hline
\multicolumn{2}{r}{\makebox[0pt][r]{\sphinxtablecontinued{continues on next page}}}\\
\endfoot

\endlastfoot

{\hyperref[\detokenize{_autosummary/quadrilaterals.parallelograms.parallelogram.Parallelogram:quadrilaterals.parallelograms.parallelogram.Parallelogram.__init__}]{\sphinxcrossref{\sphinxcode{\sphinxupquote{\_\_init\_\_}}}}}(parallel\_side\_A\_length, …{[}, …{]})
&
Initialize self.
\\
\hline
\sphinxcode{\sphinxupquote{area}}()
&
Return the area of the parallelogram.
\\
\hline
\sphinxcode{\sphinxupquote{axes\_of\_symmetry}}()
&
Return the number of axes of symmetry of the quadrilateral.
\\
\hline
\sphinxcode{\sphinxupquote{perimeter}}()
&
Return the perimeter of the quadrilateral.
\\
\hline
\end{longtable}\sphinxatlongtableend\end{savenotes}
\subsubsection*{Attributes}


\begin{savenotes}\sphinxatlongtablestart\begin{longtable}[c]{\X{1}{2}\X{1}{2}}
\hline

\endfirsthead

\multicolumn{2}{c}%
{\makebox[0pt]{\sphinxtablecontinued{\tablename\ \thetable{} \textendash{} continued from previous page}}}\\
\hline

\endhead

\hline
\multicolumn{2}{r}{\makebox[0pt][r]{\sphinxtablecontinued{continues on next page}}}\\
\endfoot

\endlastfoot

\sphinxcode{\sphinxupquote{dimensions}}
&

\\
\hline
\sphinxcode{\sphinxupquote{sides}}
&

\\
\hline
\end{longtable}\sphinxatlongtableend\end{savenotes}

\end{fulllineitems}



\section{quadrilaterals.parallelograms.rectangles.rectangle.Rectangle}
\label{\detokenize{_autosummary/quadrilaterals.parallelograms.rectangles.rectangle.Rectangle:quadrilaterals-parallelograms-rectangles-rectangle-rectangle}}\label{\detokenize{_autosummary/quadrilaterals.parallelograms.rectangles.rectangle.Rectangle::doc}}\index{Rectangle (class in quadrilaterals.parallelograms.rectangles.rectangle)@\spxentry{Rectangle}\spxextra{class in quadrilaterals.parallelograms.rectangles.rectangle}}

\begin{fulllineitems}
\phantomsection\label{\detokenize{_autosummary/quadrilaterals.parallelograms.rectangles.rectangle.Rectangle:quadrilaterals.parallelograms.rectangles.rectangle.Rectangle}}\pysiglinewithargsret{\sphinxbfcode{\sphinxupquote{class }}\sphinxcode{\sphinxupquote{quadrilaterals.parallelograms.rectangles.rectangle.}}\sphinxbfcode{\sphinxupquote{Rectangle}}}{\emph{\DUrole{n}{height}}, \emph{\DUrole{n}{width}}}{}
Base class common to all rectangles.

\begin{sphinxVerbatim}[commandchars=\\\{\}]
\PYG{g+gp}{\PYGZgt{}\PYGZgt{}\PYGZgt{} }\PYG{k+kn}{import} \PYG{n+nn}{quadrilaterals}
\PYG{g+gp}{\PYGZgt{}\PYGZgt{}\PYGZgt{} }\PYG{n}{q} \PYG{o}{=} \PYG{n}{quadrilaterals}\PYG{o}{.}\PYG{n}{Rectangle}\PYG{p}{(}\PYG{l+m+mi}{80}\PYG{p}{,} \PYG{l+m+mi}{200}\PYG{p}{)}
\end{sphinxVerbatim}
\index{\_\_init\_\_() (quadrilaterals.parallelograms.rectangles.rectangle.Rectangle method)@\spxentry{\_\_init\_\_()}\spxextra{quadrilaterals.parallelograms.rectangles.rectangle.Rectangle method}}

\begin{fulllineitems}
\phantomsection\label{\detokenize{_autosummary/quadrilaterals.parallelograms.rectangles.rectangle.Rectangle:quadrilaterals.parallelograms.rectangles.rectangle.Rectangle.__init__}}\pysiglinewithargsret{\sphinxbfcode{\sphinxupquote{\_\_init\_\_}}}{\emph{\DUrole{n}{height}}, \emph{\DUrole{n}{width}}}{}
Initialize self.  See help(type(self)) for accurate signature.

\end{fulllineitems}

\subsubsection*{Methods}


\begin{savenotes}\sphinxatlongtablestart\begin{longtable}[c]{\X{1}{2}\X{1}{2}}
\hline

\endfirsthead

\multicolumn{2}{c}%
{\makebox[0pt]{\sphinxtablecontinued{\tablename\ \thetable{} \textendash{} continued from previous page}}}\\
\hline

\endhead

\hline
\multicolumn{2}{r}{\makebox[0pt][r]{\sphinxtablecontinued{continues on next page}}}\\
\endfoot

\endlastfoot

{\hyperref[\detokenize{_autosummary/quadrilaterals.parallelograms.rectangles.rectangle.Rectangle:quadrilaterals.parallelograms.rectangles.rectangle.Rectangle.__init__}]{\sphinxcrossref{\sphinxcode{\sphinxupquote{\_\_init\_\_}}}}}(height, width)
&
Initialize self.
\\
\hline
\sphinxcode{\sphinxupquote{area}}()
&
Return the area of the rectangle.
\\
\hline
\sphinxcode{\sphinxupquote{axes\_of\_symmetry}}()
&
Return the number of axes of symmetry of the quadrilateral.
\\
\hline
\sphinxcode{\sphinxupquote{perimeter}}()
&
Return the perimeter of the quadrilateral.
\\
\hline
\end{longtable}\sphinxatlongtableend\end{savenotes}
\subsubsection*{Attributes}


\begin{savenotes}\sphinxatlongtablestart\begin{longtable}[c]{\X{1}{2}\X{1}{2}}
\hline

\endfirsthead

\multicolumn{2}{c}%
{\makebox[0pt]{\sphinxtablecontinued{\tablename\ \thetable{} \textendash{} continued from previous page}}}\\
\hline

\endhead

\hline
\multicolumn{2}{r}{\makebox[0pt][r]{\sphinxtablecontinued{continues on next page}}}\\
\endfoot

\endlastfoot

\sphinxcode{\sphinxupquote{dimensions}}
&

\\
\hline
\sphinxcode{\sphinxupquote{sides}}
&

\\
\hline
\end{longtable}\sphinxatlongtableend\end{savenotes}

\end{fulllineitems}



\section{quadrilaterals.parallelograms.rectangles.square.Square}
\label{\detokenize{_autosummary/quadrilaterals.parallelograms.rectangles.square.Square:quadrilaterals-parallelograms-rectangles-square-square}}\label{\detokenize{_autosummary/quadrilaterals.parallelograms.rectangles.square.Square::doc}}\index{Square (class in quadrilaterals.parallelograms.rectangles.square)@\spxentry{Square}\spxextra{class in quadrilaterals.parallelograms.rectangles.square}}

\begin{fulllineitems}
\phantomsection\label{\detokenize{_autosummary/quadrilaterals.parallelograms.rectangles.square.Square:quadrilaterals.parallelograms.rectangles.square.Square}}\pysiglinewithargsret{\sphinxbfcode{\sphinxupquote{class }}\sphinxcode{\sphinxupquote{quadrilaterals.parallelograms.rectangles.square.}}\sphinxbfcode{\sphinxupquote{Square}}}{\emph{\DUrole{n}{side\_length}}}{}
Base class common to all squares.

\begin{sphinxVerbatim}[commandchars=\\\{\}]
\PYG{g+gp}{\PYGZgt{}\PYGZgt{}\PYGZgt{} }\PYG{k+kn}{import} \PYG{n+nn}{quadrilaterals}
\PYG{g+gp}{\PYGZgt{}\PYGZgt{}\PYGZgt{} }\PYG{n}{q} \PYG{o}{=} \PYG{n}{quadrilaterals}\PYG{o}{.}\PYG{n}{Square}\PYG{p}{(}\PYG{l+m+mf}{0.2}\PYG{p}{)}
\end{sphinxVerbatim}
\index{\_\_init\_\_() (quadrilaterals.parallelograms.rectangles.square.Square method)@\spxentry{\_\_init\_\_()}\spxextra{quadrilaterals.parallelograms.rectangles.square.Square method}}

\begin{fulllineitems}
\phantomsection\label{\detokenize{_autosummary/quadrilaterals.parallelograms.rectangles.square.Square:quadrilaterals.parallelograms.rectangles.square.Square.__init__}}\pysiglinewithargsret{\sphinxbfcode{\sphinxupquote{\_\_init\_\_}}}{\emph{\DUrole{n}{side\_length}}}{}
Initialize self.  See help(type(self)) for accurate signature.

\end{fulllineitems}

\subsubsection*{Methods}


\begin{savenotes}\sphinxatlongtablestart\begin{longtable}[c]{\X{1}{2}\X{1}{2}}
\hline

\endfirsthead

\multicolumn{2}{c}%
{\makebox[0pt]{\sphinxtablecontinued{\tablename\ \thetable{} \textendash{} continued from previous page}}}\\
\hline

\endhead

\hline
\multicolumn{2}{r}{\makebox[0pt][r]{\sphinxtablecontinued{continues on next page}}}\\
\endfoot

\endlastfoot

{\hyperref[\detokenize{_autosummary/quadrilaterals.parallelograms.rectangles.square.Square:quadrilaterals.parallelograms.rectangles.square.Square.__init__}]{\sphinxcrossref{\sphinxcode{\sphinxupquote{\_\_init\_\_}}}}}(side\_length)
&
Initialize self.
\\
\hline
\sphinxcode{\sphinxupquote{area}}()
&
Return the area of the square.
\\
\hline
\sphinxcode{\sphinxupquote{axes\_of\_symmetry}}()
&
Return the number of axes of symmetry of the quadrilateral.
\\
\hline
\sphinxcode{\sphinxupquote{perimeter}}()
&
Return the perimeter of the quadrilateral.
\\
\hline
\end{longtable}\sphinxatlongtableend\end{savenotes}
\subsubsection*{Attributes}


\begin{savenotes}\sphinxatlongtablestart\begin{longtable}[c]{\X{1}{2}\X{1}{2}}
\hline

\endfirsthead

\multicolumn{2}{c}%
{\makebox[0pt]{\sphinxtablecontinued{\tablename\ \thetable{} \textendash{} continued from previous page}}}\\
\hline

\endhead

\hline
\multicolumn{2}{r}{\makebox[0pt][r]{\sphinxtablecontinued{continues on next page}}}\\
\endfoot

\endlastfoot

\sphinxcode{\sphinxupquote{dimensions}}
&

\\
\hline
\sphinxcode{\sphinxupquote{sides}}
&

\\
\hline
\end{longtable}\sphinxatlongtableend\end{savenotes}

\end{fulllineitems}



\section{quadrilaterals.trapezia.trapezium.Trapezium}
\label{\detokenize{_autosummary/quadrilaterals.trapezia.trapezium.Trapezium:quadrilaterals-trapezia-trapezium-trapezium}}\label{\detokenize{_autosummary/quadrilaterals.trapezia.trapezium.Trapezium::doc}}\index{Trapezium (class in quadrilaterals.trapezia.trapezium)@\spxentry{Trapezium}\spxextra{class in quadrilaterals.trapezia.trapezium}}

\begin{fulllineitems}
\phantomsection\label{\detokenize{_autosummary/quadrilaterals.trapezia.trapezium.Trapezium:quadrilaterals.trapezia.trapezium.Trapezium}}\pysiglinewithargsret{\sphinxbfcode{\sphinxupquote{class }}\sphinxcode{\sphinxupquote{quadrilaterals.trapezia.trapezium.}}\sphinxbfcode{\sphinxupquote{Trapezium}}}{\emph{\DUrole{n}{parallel\_side\_A\_length}}, \emph{\DUrole{n}{parallel\_side\_B\_length}}, \emph{\DUrole{n}{non\_parallel\_side\_A\_length}}, \emph{\DUrole{n}{non\_parallel\_side\_B\_length}}, \emph{\DUrole{n}{height}\DUrole{o}{=}\DUrole{default_value}{None}}}{}
Base class common to all trapezia.

\begin{sphinxVerbatim}[commandchars=\\\{\}]
\PYG{g+gp}{\PYGZgt{}\PYGZgt{}\PYGZgt{} }\PYG{n}{q} \PYG{o}{=} \PYG{n}{quadrilaterals}\PYG{o}{.}\PYG{n}{Trapezium}\PYG{p}{(}\PYG{l+m+mi}{5}\PYG{p}{,} \PYG{l+m+mi}{7}\PYG{p}{,} \PYG{l+m+mi}{5}\PYG{p}{,} \PYG{l+m+mi}{4}\PYG{p}{,} \PYG{l+m+mi}{4}\PYG{p}{)}
\end{sphinxVerbatim}
\index{\_\_init\_\_() (quadrilaterals.trapezia.trapezium.Trapezium method)@\spxentry{\_\_init\_\_()}\spxextra{quadrilaterals.trapezia.trapezium.Trapezium method}}

\begin{fulllineitems}
\phantomsection\label{\detokenize{_autosummary/quadrilaterals.trapezia.trapezium.Trapezium:quadrilaterals.trapezia.trapezium.Trapezium.__init__}}\pysiglinewithargsret{\sphinxbfcode{\sphinxupquote{\_\_init\_\_}}}{\emph{\DUrole{n}{parallel\_side\_A\_length}}, \emph{\DUrole{n}{parallel\_side\_B\_length}}, \emph{\DUrole{n}{non\_parallel\_side\_A\_length}}, \emph{\DUrole{n}{non\_parallel\_side\_B\_length}}, \emph{\DUrole{n}{height}\DUrole{o}{=}\DUrole{default_value}{None}}}{}
Initialize self.  See help(type(self)) for accurate signature.

\end{fulllineitems}

\subsubsection*{Methods}


\begin{savenotes}\sphinxatlongtablestart\begin{longtable}[c]{\X{1}{2}\X{1}{2}}
\hline

\endfirsthead

\multicolumn{2}{c}%
{\makebox[0pt]{\sphinxtablecontinued{\tablename\ \thetable{} \textendash{} continued from previous page}}}\\
\hline

\endhead

\hline
\multicolumn{2}{r}{\makebox[0pt][r]{\sphinxtablecontinued{continues on next page}}}\\
\endfoot

\endlastfoot

{\hyperref[\detokenize{_autosummary/quadrilaterals.trapezia.trapezium.Trapezium:quadrilaterals.trapezia.trapezium.Trapezium.__init__}]{\sphinxcrossref{\sphinxcode{\sphinxupquote{\_\_init\_\_}}}}}(parallel\_side\_A\_length, …{[}, height{]})
&
Initialize self.
\\
\hline
\sphinxcode{\sphinxupquote{area}}()
&
Return the area of the trapezium.
\\
\hline
\sphinxcode{\sphinxupquote{axes\_of\_symmetry}}()
&
Return the number of axes of symmetry of the quadrilateral.
\\
\hline
\sphinxcode{\sphinxupquote{perimeter}}()
&
Return the perimeter of the quadrilateral.
\\
\hline
\end{longtable}\sphinxatlongtableend\end{savenotes}
\subsubsection*{Attributes}


\begin{savenotes}\sphinxatlongtablestart\begin{longtable}[c]{\X{1}{2}\X{1}{2}}
\hline

\endfirsthead

\multicolumn{2}{c}%
{\makebox[0pt]{\sphinxtablecontinued{\tablename\ \thetable{} \textendash{} continued from previous page}}}\\
\hline

\endhead

\hline
\multicolumn{2}{r}{\makebox[0pt][r]{\sphinxtablecontinued{continues on next page}}}\\
\endfoot

\endlastfoot

\sphinxcode{\sphinxupquote{dimensions}}
&

\\
\hline
\sphinxcode{\sphinxupquote{sides}}
&

\\
\hline
\end{longtable}\sphinxatlongtableend\end{savenotes}

\end{fulllineitems}



\section{quadrilaterals.trapezia.isosceles\_trapezia.isosceles\_trapezium.IsoscelesTrapezium}
\label{\detokenize{_autosummary/quadrilaterals.trapezia.isosceles_trapezia.isosceles_trapezium.IsoscelesTrapezium:quadrilaterals-trapezia-isosceles-trapezia-isosceles-trapezium-isoscelestrapezium}}\label{\detokenize{_autosummary/quadrilaterals.trapezia.isosceles_trapezia.isosceles_trapezium.IsoscelesTrapezium::doc}}\index{IsoscelesTrapezium (class in quadrilaterals.trapezia.isosceles\_trapezia.isosceles\_trapezium)@\spxentry{IsoscelesTrapezium}\spxextra{class in quadrilaterals.trapezia.isosceles\_trapezia.isosceles\_trapezium}}

\begin{fulllineitems}
\phantomsection\label{\detokenize{_autosummary/quadrilaterals.trapezia.isosceles_trapezia.isosceles_trapezium.IsoscelesTrapezium:quadrilaterals.trapezia.isosceles_trapezia.isosceles_trapezium.IsoscelesTrapezium}}\pysiglinewithargsret{\sphinxbfcode{\sphinxupquote{class }}\sphinxcode{\sphinxupquote{quadrilaterals.trapezia.isosceles\_trapezia.isosceles\_trapezium.}}\sphinxbfcode{\sphinxupquote{IsoscelesTrapezium}}}{\emph{\DUrole{n}{parallel\_side\_A\_length}}, \emph{\DUrole{n}{parallel\_side\_B\_length}}, \emph{\DUrole{n}{non\_parallel\_sides\_length}}}{}
Base class common to all isosceles trapezia.

\begin{sphinxVerbatim}[commandchars=\\\{\}]
\PYG{g+gp}{\PYGZgt{}\PYGZgt{}\PYGZgt{} }\PYG{n}{q} \PYG{o}{=} \PYG{n}{quadrilaterals}\PYG{o}{.}\PYG{n}{IsoscelesTrapezium}\PYG{p}{(}\PYG{l+m+mi}{4}\PYG{p}{,} \PYG{l+m+mi}{5}\PYG{p}{,} \PYG{l+m+mi}{6}\PYG{p}{)}
\end{sphinxVerbatim}
\index{\_\_init\_\_() (quadrilaterals.trapezia.isosceles\_trapezia.isosceles\_trapezium.IsoscelesTrapezium method)@\spxentry{\_\_init\_\_()}\spxextra{quadrilaterals.trapezia.isosceles\_trapezia.isosceles\_trapezium.IsoscelesTrapezium method}}

\begin{fulllineitems}
\phantomsection\label{\detokenize{_autosummary/quadrilaterals.trapezia.isosceles_trapezia.isosceles_trapezium.IsoscelesTrapezium:quadrilaterals.trapezia.isosceles_trapezia.isosceles_trapezium.IsoscelesTrapezium.__init__}}\pysiglinewithargsret{\sphinxbfcode{\sphinxupquote{\_\_init\_\_}}}{\emph{\DUrole{n}{parallel\_side\_A\_length}}, \emph{\DUrole{n}{parallel\_side\_B\_length}}, \emph{\DUrole{n}{non\_parallel\_sides\_length}}}{}
Initialize self.  See help(type(self)) for accurate signature.

\end{fulllineitems}

\subsubsection*{Methods}


\begin{savenotes}\sphinxatlongtablestart\begin{longtable}[c]{\X{1}{2}\X{1}{2}}
\hline

\endfirsthead

\multicolumn{2}{c}%
{\makebox[0pt]{\sphinxtablecontinued{\tablename\ \thetable{} \textendash{} continued from previous page}}}\\
\hline

\endhead

\hline
\multicolumn{2}{r}{\makebox[0pt][r]{\sphinxtablecontinued{continues on next page}}}\\
\endfoot

\endlastfoot

{\hyperref[\detokenize{_autosummary/quadrilaterals.trapezia.isosceles_trapezia.isosceles_trapezium.IsoscelesTrapezium:quadrilaterals.trapezia.isosceles_trapezia.isosceles_trapezium.IsoscelesTrapezium.__init__}]{\sphinxcrossref{\sphinxcode{\sphinxupquote{\_\_init\_\_}}}}}(parallel\_side\_A\_length, …)
&
Initialize self.
\\
\hline
\sphinxcode{\sphinxupquote{area}}()
&
Return the area of the isosceles trapezium.
\\
\hline
\sphinxcode{\sphinxupquote{axes\_of\_symmetry}}()
&
Return the number of axes of symmetry of the quadrilateral.
\\
\hline
\sphinxcode{\sphinxupquote{brahmagupta\_formula}}(a, b, c, d)
&
Calculate area from side lengths for cyclic quadrilaterals only.
\\
\hline
\sphinxcode{\sphinxupquote{perimeter}}()
&
Return the perimeter of the quadrilateral.
\\
\hline
\end{longtable}\sphinxatlongtableend\end{savenotes}
\subsubsection*{Attributes}


\begin{savenotes}\sphinxatlongtablestart\begin{longtable}[c]{\X{1}{2}\X{1}{2}}
\hline

\endfirsthead

\multicolumn{2}{c}%
{\makebox[0pt]{\sphinxtablecontinued{\tablename\ \thetable{} \textendash{} continued from previous page}}}\\
\hline

\endhead

\hline
\multicolumn{2}{r}{\makebox[0pt][r]{\sphinxtablecontinued{continues on next page}}}\\
\endfoot

\endlastfoot

\sphinxcode{\sphinxupquote{dimensions}}
&

\\
\hline
\sphinxcode{\sphinxupquote{sides}}
&

\\
\hline
\end{longtable}\sphinxatlongtableend\end{savenotes}

\end{fulllineitems}



\chapter{Indices and tables}
\label{\detokenize{index:indices-and-tables}}\begin{itemize}
\item {} 
\DUrole{xref,std,std-ref}{genindex}

\item {} 
\DUrole{xref,std,std-ref}{modindex}

\item {} 
\DUrole{xref,std,std-ref}{search}

\end{itemize}



\renewcommand{\indexname}{Index}
\printindex
\end{document}